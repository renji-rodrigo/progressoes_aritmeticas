\documentclass[12pt]{book}
\usepackage{amsmath}
\usepackage{gensymb}
\usepackage[pdftex]{hyperref}
\usepackage{hyperref}
\usepackage[table,xcdraw]{xcolor}
\usepackage{lmodern}
\usepackage{titlesec}
\usepackage{microtype}
\usepackage{scalerel,xparse}
\usepackage{xcolor}
\usepackage{enumitem}
\usepackage{amstext}
\usepackage[T1]{fontenc}
\usepackage[brazil,brazilian]{babel}
%\usepackage{showkeys}
\usepackage{color}
\NewDocumentCommand\bird{}{
    \includegraphics[scale=0.14]{../../../LaTeX/Imagens/house.jpg}
}
\usepackage{caption}
\def\pre{\textcolor[rgb]{1,0.501961,0}{\textbf{Pr�mio:\,}}}
\def\so{\text{ \textbf{Solu��o:\,}}}
\def\res{\text{ \textbf{Resposta:\,}}}
\def\reso{\text{\textcolor[rgb]{0,0.501961,0.501961}{\textbf{Resolva a quest�o: }} }}
\def\lef{\left\{}
\def\lp{\lfloor}
\def\rp{\rfloor}
\def\mb{\text}
\def\bra{\langle}
\def\ket{\rangle}
\def\autor{Rodrigo Carlos Silva de Lima\\ \href{rodrigo.uff.math@gmail.com}{rodrigo.uff.math@gmail.com}\\\href{https://apoia.se/atelierdesonhos}{https://apoia.se/atelierdesonhos}}
\def\f{\varphi}
\def\li{\lim\limits_{n\rar \F}}
\def\de{\displaystyle}
\def\eq{\Leftrightarrow}
\def\p{\partial}
\def\va{\emptyset}
\def\es{\qquad}
\def\ta{\theta}
\def\ga{\gamma}
\def\g{^{\circ}}
\def\tr{\triangle}
\def\sm{\small}
\def\fs{\footnotesize}
\def\ns{\normalsize}
\def\T{\times}
\def\mc{$\text{ modificador de carisma } $}
\def\I{\item}
\def\m{\mu}
\def\oulo{\lor}
\def\elo{\wedge}
\def\F{\infty}
\def\C{\cdot}
\def\le{\left}
\def\ri{\right}
\def\n{\nu}
\def\N{\nabla}
\def\Om{\Omega}
\def\om{\omega}
\def\na{{\mathbb{N}}}
\def\a{\alpha}
\def\lf{\lfloor}
\def\rf{\rfloor}
\def\e{\epsilon}
\def\ve{\varepsilon}
\def\ex{\exists}
\def\b{\beta}
\def\Ra{\Rightarrow}
\def\re{\mathbb{R}}
\def\s{\subset}
\def\rar{\rightarrow}
\def\D{\Delta}
\def\fa{\forall \;}
\usepackage{amssymb}
\usepackage{amsthm}
\usepackage{amsmath}
%\newtheorem{ques}{\textcolor[rgb]{0.00,0.25,0.25}{ {\Large \Writinghand}Quest�o}}
\usepackage{bookman}
\usepackage{gfsartemisia-euler}
\usepackage{eulervm}
\usepackage{pst-all}
\usepackage{pstricks-add}
\usepackage{pstricks}
\usepackage{auto-pst-pdf}
\usepackage{tikz}
\usepackage{shadethm}
\usepackage{mdframed}
\usepackage{marvosym}
\usepackage{wasysym}
\usepackage{tikzsymbols}
\usepackage{amssymb,amsmath}
\usepackage{color}
\usepackage{textcomp}
\usepackage{cancel}
\makeatletter
\DeclareFontFamily{U}{tipa}{}
\DeclareFontShape{U}{tipa}{m}{n}{<->tipa10}{}
\newcommand{\arc@char}{{\usefont{U}{tipa}{m}{n}\symbol{62}}}%

\newcommand{\arc}[1]{\mathpalette\arc@arc{#1}}

\newcommand{\arc@arc}[2]{%
  \sbox0{$\m@th#1#2$}%
  \vbox{
    \hbox{\resizebox{\wd0}{\height}{\arc@char}}
    \nointerlineskip
    \box0
  }%
}
\makeatother
\usepackage{graphicx}
\newtheorem{obs}{ {\Large \Emailct} Observa��o}
\newtheorem{dica}{ {\Large \Emailct} Dica de Solu��o}
\usepackage{float}
\usepackage[colorlinks,linkcolor=blue,hyperindex]{hyperref}
\newmdtheoremenv[
  hidealllines=true,
  roundcorner=12pt,
  innerleftmargin=8pt,%
  innerrightmargin=8pt,%
  innertopmargin=0pt,%
  innerbottommargin=2pt,%
  backgroundcolor=gray!0,%
  skipbelow=\baselineskip,%
  skipabove=\baselineskip,
  shadow=true,
  topline = true, rightline = true,
leftline = true, bottomline = true]{defi}{ \textcolor[RGB]{0,82,155}{ \bird  Defini��o }}
\renewcommand{\proof}{ \textbf{\textcolor[RGB]{0,82,155}{ { \Aquarius}  Demonstra��o}}.}
\hypersetup{colorlinks=true, raiselinks=false, anchorcolor=green,
linkcolor=blue, %%% cor do tableofcontents, \ref, \footnote, etc
citecolor=blue, %%% cor do \cite
urlcolor=blue, %%% cor do \url e \href
}
\newmdtheoremenv[
  hidealllines=true,
  roundcorner=12pt,
  innerleftmargin=8pt,%
  innerrightmargin=8pt,%
  innertopmargin=0pt,%
  innerbottommargin=2pt,%
  backgroundcolor=gray!0,%
  skipbelow=\baselineskip,%
  skipabove=\baselineskip,
  shadow=false,
  topline = true, rightline = true,
leftline = true, bottomline = true]{cor}{\textcolor[RGB]{0,128,128}{ {\large \leftmoon} Corol�rio}}
\newmdtheoremenv[
  hidealllines=true,
  roundcorner=12pt,
  innerleftmargin=8pt,%
  innerrightmargin=8pt,%
  innertopmargin=0pt,%
  innerbottommargin=2pt,%
  backgroundcolor=gray!0,%
  skipbelow=\baselineskip,%
  skipabove=\baselineskip,
  shadow=false,
  topline = true, rightline = true,
leftline = true, bottomline = true]{exem}{\textcolor[RGB]{0,128,128}{{ \LARGE \Pointinghand}  Exemplo}}
\newtheorem{lema}{ {$\clubsuit$} Lema}
\newtheorem{conje}{\textcolor[RGB]{0,82,155}{ {\Large \Writinghand  } Conjectura}}
\newtheorem{axioma}{\textcolor[RGB]{0,82,155}{{\Celtcross} Axioma}}
\newmdtheoremenv[
  hidealllines=true,
  roundcorner=12pt,
  innerleftmargin=8pt,%
  innerrightmargin=8pt,%
  innertopmargin=0pt,%
  innerbottommargin=2pt,%
  backgroundcolor=gray!12,%
  skipbelow=\baselineskip,%
  skipabove=\baselineskip,
  shadow=false,
  topline = true, rightline = false,
leftline = false, bottomline = true]{propi}{ \textcolor[RGB]{0,0,0}{ {\Large \Writinghand  } Propriedade} }
\newmdtheoremenv[
  hidealllines=true,
  roundcorner=12pt,
  innerleftmargin=8pt,%
  innerrightmargin=8pt,%
  innertopmargin=0pt,%
  innerbottommargin=2pt,%
  backgroundcolor=gray!12,%
  skipbelow=\baselineskip,%
  skipabove=\baselineskip,
  shadow=false,
  topline = true, rightline = false,
leftline = false, bottomline = true]{descri}{ \textcolor[rgb]{0.00,0.25,0.25}{ {\Large \Writinghand  } Descri��o} }
\newmdtheoremenv[
  hidealllines=true,
  roundcorner=12pt,
  innerleftmargin=8pt,%
  innerrightmargin=8pt,%
  innertopmargin=0pt,%
  innerbottommargin=2pt,%
  backgroundcolor=gray!12,%
  skipbelow=\baselineskip,%
  skipabove=\baselineskip,
  shadow=false,
  topline = true, rightline = false,
leftline = false, bottomline = true]{ops}{ \textcolor[rgb]{0.00,0.25,0.25}{ {\Large \Writinghand  } Op��es} }
\newmdtheoremenv[
  hidealllines=true,
  roundcorner=12pt,
  innerleftmargin=8pt,%
  innerrightmargin=8pt,%
  innertopmargin=0pt,%
  innerbottommargin=2pt,%
  backgroundcolor=gray!12,%
  skipbelow=\baselineskip,%
  skipabove=\baselineskip,
  shadow=false,
  topline = true, rightline = false,
leftline = false, bottomline = true]{cen}{ \textcolor[rgb]{0.00,0.25,0.25}{ {\Large \Writinghand  } Cen�rio} }
\newmdtheoremenv[
  hidealllines=true,
  roundcorner=12pt,
  innerleftmargin=8pt,%
  innerrightmargin=8pt,%
  innertopmargin=0pt,%
  innerbottommargin=2pt,%
  backgroundcolor=gray!0,%
  skipbelow=\baselineskip,%
  skipabove=\baselineskip,
  shadow=true,
  topline = true, rightline = true,
leftline = true, bottomline = true]{hm}{ \textcolor[rgb]{0.00,0.25,0.50}{ {\Large\Mundus} Hist�ria }}
\newmdtheoremenv[
  hidealllines=true,
  roundcorner=12pt,
  innerleftmargin=8pt,%
  innerrightmargin=8pt,%
  innertopmargin=0pt,%
  innerbottommargin=2pt,%
  backgroundcolor=gray!0,%
  skipbelow=\baselineskip,%
  skipabove=\baselineskip,
  shadow=true,
  topline = true, rightline = true,
leftline = true, bottomline = true]{comb}{ \textcolor[rgb]{0.00,0.25,0.50}{ {\Large\Mundus} Combate }}
\usepackage{makeidx}
\def\d{\displaystyle}
\def\de{\displaystyle}
\usepackage[T1]{fontenc}
\usepackage[brazilian]{babel}
\usepackage{amssymb}
\usepackage{amsmath}
\usepackage{color}
\usepackage{graphicx}
\newmdtheoremenv[
  hidealllines=true,
  roundcorner=12pt,
  innerleftmargin=8pt,%
  innerrightmargin=8pt,%
  innertopmargin=0pt,%
  innerbottommargin=2pt,%
  backgroundcolor=gray!12,%
  skipbelow=\baselineskip,%
  skipabove=\baselineskip,
  shadow=false,
  topline = true, rightline = false,
leftline = false, bottomline = true]{teo}{\textcolor[RGB]{0,0,0}{ {\Large \Writinghand  }Teorema }}
\newtheorem{ques}{\textcolor[RGB]{0,82,155}{ {\Large \Writinghand  } Quest�o}}
\renewcommand{\rmdefault}{@georgia}
\renewcommand{\rmdefault}{@ink@free}
\renewcommand{\rmdefault}{@georgia}
\renewcommand{\rmdefault}{artemisia}
\renewcommand{\rmdefault}{@georgia}
\renewcommand{\rmdefault}{artemisia}
\renewcommand{\rmdefault}{@georgia}
\renewcommand{\rmdefault}{fac}
\renewcommand{\rmdefault}{Alegreya-LF}
%Comandos
\newcommand*\circled[1]{\tikz[baseline=(char.base)]{
            \node[shape=circle,draw,inner sep=2pt] (char) {#1};}}
            
